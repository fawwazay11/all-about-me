% Options for packages loaded elsewhere
% Options for packages loaded elsewhere
\PassOptionsToPackage{unicode}{hyperref}
\PassOptionsToPackage{hyphens}{url}
\PassOptionsToPackage{dvipsnames,svgnames,x11names}{xcolor}
%
\documentclass[
  letterpaper,
  DIV=11,
  numbers=noendperiod]{scrreprt}
\usepackage{xcolor}
\usepackage{amsmath,amssymb}
\setcounter{secnumdepth}{5}
\usepackage{iftex}
\ifPDFTeX
  \usepackage[T1]{fontenc}
  \usepackage[utf8]{inputenc}
  \usepackage{textcomp} % provide euro and other symbols
\else % if luatex or xetex
  \usepackage{unicode-math} % this also loads fontspec
  \defaultfontfeatures{Scale=MatchLowercase}
  \defaultfontfeatures[\rmfamily]{Ligatures=TeX,Scale=1}
\fi
\usepackage{lmodern}
\ifPDFTeX\else
  % xetex/luatex font selection
\fi
% Use upquote if available, for straight quotes in verbatim environments
\IfFileExists{upquote.sty}{\usepackage{upquote}}{}
\IfFileExists{microtype.sty}{% use microtype if available
  \usepackage[]{microtype}
  \UseMicrotypeSet[protrusion]{basicmath} % disable protrusion for tt fonts
}{}
\makeatletter
\@ifundefined{KOMAClassName}{% if non-KOMA class
  \IfFileExists{parskip.sty}{%
    \usepackage{parskip}
  }{% else
    \setlength{\parindent}{0pt}
    \setlength{\parskip}{6pt plus 2pt minus 1pt}}
}{% if KOMA class
  \KOMAoptions{parskip=half}}
\makeatother
% Make \paragraph and \subparagraph free-standing
\makeatletter
\ifx\paragraph\undefined\else
  \let\oldparagraph\paragraph
  \renewcommand{\paragraph}{
    \@ifstar
      \xxxParagraphStar
      \xxxParagraphNoStar
  }
  \newcommand{\xxxParagraphStar}[1]{\oldparagraph*{#1}\mbox{}}
  \newcommand{\xxxParagraphNoStar}[1]{\oldparagraph{#1}\mbox{}}
\fi
\ifx\subparagraph\undefined\else
  \let\oldsubparagraph\subparagraph
  \renewcommand{\subparagraph}{
    \@ifstar
      \xxxSubParagraphStar
      \xxxSubParagraphNoStar
  }
  \newcommand{\xxxSubParagraphStar}[1]{\oldsubparagraph*{#1}\mbox{}}
  \newcommand{\xxxSubParagraphNoStar}[1]{\oldsubparagraph{#1}\mbox{}}
\fi
\makeatother


\usepackage{longtable,booktabs,array}
\usepackage{calc} % for calculating minipage widths
% Correct order of tables after \paragraph or \subparagraph
\usepackage{etoolbox}
\makeatletter
\patchcmd\longtable{\par}{\if@noskipsec\mbox{}\fi\par}{}{}
\makeatother
% Allow footnotes in longtable head/foot
\IfFileExists{footnotehyper.sty}{\usepackage{footnotehyper}}{\usepackage{footnote}}
\makesavenoteenv{longtable}
\usepackage{graphicx}
\makeatletter
\newsavebox\pandoc@box
\newcommand*\pandocbounded[1]{% scales image to fit in text height/width
  \sbox\pandoc@box{#1}%
  \Gscale@div\@tempa{\textheight}{\dimexpr\ht\pandoc@box+\dp\pandoc@box\relax}%
  \Gscale@div\@tempb{\linewidth}{\wd\pandoc@box}%
  \ifdim\@tempb\p@<\@tempa\p@\let\@tempa\@tempb\fi% select the smaller of both
  \ifdim\@tempa\p@<\p@\scalebox{\@tempa}{\usebox\pandoc@box}%
  \else\usebox{\pandoc@box}%
  \fi%
}
% Set default figure placement to htbp
\def\fps@figure{htbp}
\makeatother





\setlength{\emergencystretch}{3em} % prevent overfull lines

\providecommand{\tightlist}{%
  \setlength{\itemsep}{0pt}\setlength{\parskip}{0pt}}



 


\KOMAoption{captions}{tableheading}
\makeatletter
\@ifpackageloaded{bookmark}{}{\usepackage{bookmark}}
\makeatother
\makeatletter
\@ifpackageloaded{caption}{}{\usepackage{caption}}
\AtBeginDocument{%
\ifdefined\contentsname
  \renewcommand*\contentsname{Table of contents}
\else
  \newcommand\contentsname{Table of contents}
\fi
\ifdefined\listfigurename
  \renewcommand*\listfigurename{List of Figures}
\else
  \newcommand\listfigurename{List of Figures}
\fi
\ifdefined\listtablename
  \renewcommand*\listtablename{List of Tables}
\else
  \newcommand\listtablename{List of Tables}
\fi
\ifdefined\figurename
  \renewcommand*\figurename{Figure}
\else
  \newcommand\figurename{Figure}
\fi
\ifdefined\tablename
  \renewcommand*\tablename{Table}
\else
  \newcommand\tablename{Table}
\fi
}
\@ifpackageloaded{float}{}{\usepackage{float}}
\floatstyle{ruled}
\@ifundefined{c@chapter}{\newfloat{codelisting}{h}{lop}}{\newfloat{codelisting}{h}{lop}[chapter]}
\floatname{codelisting}{Listing}
\newcommand*\listoflistings{\listof{codelisting}{List of Listings}}
\makeatother
\makeatletter
\makeatother
\makeatletter
\@ifpackageloaded{caption}{}{\usepackage{caption}}
\@ifpackageloaded{subcaption}{}{\usepackage{subcaption}}
\makeatother
\usepackage{bookmark}
\IfFileExists{xurl.sty}{\usepackage{xurl}}{} % add URL line breaks if available
\urlstyle{same}
\hypersetup{
  pdftitle={Fawwaz Aydin Mustofa},
  pdfauthor={18222109 Fawwaz Aydin Mustofa},
  colorlinks=true,
  linkcolor={blue},
  filecolor={Maroon},
  citecolor={Blue},
  urlcolor={Blue},
  pdfcreator={LaTeX via pandoc}}


\title{Fawwaz Aydin Mustofa}
\usepackage{etoolbox}
\makeatletter
\providecommand{\subtitle}[1]{% add subtitle to \maketitle
  \apptocmd{\@title}{\par {\large #1 \par}}{}{}
}
\makeatother
\subtitle{Portfolio Asesmen II-2100 KIPP}
\author{18222109 Fawwaz Aydin Mustofa}
\date{2025-10-20}
\begin{document}
\maketitle

\renewcommand*\contentsname{Table of contents}
{
\hypersetup{linkcolor=}
\setcounter{tocdepth}{2}
\tableofcontents
}

\bookmarksetup{startatroot}

\chapter*{Selamat Datang}\label{selamat-datang}
\addcontentsline{toc}{chapter}{Selamat Datang}

\markboth{Selamat Datang}{Selamat Datang}

Halo Aku Fawwaz Aydin Mustofa. Seorang mahasiswa STI ITB 2022

\bookmarksetup{startatroot}

\chapter{UTS-1 All About Me}\label{uts-1-all-about-me}

\section{\texorpdfstring{\textbf{Introduction}}{Introduction}}\label{introduction}

Halo semua! kenalan dulu yok, aku Fawwaz Aydin Mustofa, mahasiswa
tingkat akhir Sistem dan Teknologi Informasi di ITB. Aku orangnya
cenderung pendiem kalau di tempat baru, tapi begitu udah kenal, biasanya
malah jadi banyak cerita. Di luar kuliah, aku suka main game, ngobrol
bareng teman di Discord, baca manga, atau nonton anime. Fun fact, aku
punya saudara kembar yang kuliah di ITS jurusan Teknik Informatika.

Selama kuliah, aku sadar kalau aku tipe orang yang suka ngulik sesuatu
sampai ngerti kenapa itu bisa jalan. Kadang rasa penasaranku malah bawa
aku ke hal-hal yang nggak ada hubungannya sama tugas, tapi justru di
situ aku nemu banyak hal menarik. Aku juga orang yang cukup peka sama
sekitar --- aku lebih nyaman mendengarkan dulu, baru bicara kalau aku
yakin bisa bantu. Dari beberapa pengalaman pribadi, aku belajar banyak
tentang pentingnya memahami diri sendiri dan nggak terus hidup di bawah
ekspektasi orang lain. Sekarang, aku lagi berusaha jadi versi terbaik
dari diriku sendiri, yang tetap tulus dan jujur sama apa yang aku
yakini.

Ke depannya, aku pengin terus tumbuh di lingkungan yang bisa ngajarin
aku hal baru dan di mana aku bisa ngasih dampak positif lewat teknologi.
Buatku, hal paling keren dari dunia IT itu bukan cuma tentang seberapa
canggih sistemnya, tapi seberapa besar manfaatnya buat orang lain.

\bookmarksetup{startatroot}

\chapter{UTS-2 My Songs for You}\label{uts-2-my-songs-for-you}

Song: Halfway There - Big Time Rush

\url{https://youtu.be/v-87vtFhCqI?si=UXjaiaRZ3zpD3FCh}

Lagu ini saya tujukan untuk teman - teman saya yg sedang berjuang dan
juga untuk saya sendiri. lagu ini membawa pesan bahwa hidup itu tidak
sesederhana keinginan, jika kita tidak mencoba kita tidak akan jatuh
namun, kita juga tidak akan pernah melangkah lebih jauh. lagu ini juga
membawa pesan kebersamaan seperti pada lirik ``so we take what comes and
we keep on going, leaning on each other's shoulders Then we turn around,
And see we've come so far somehow'' kita menerima apapun yg datang dan
terus melangkah bersama, saling menopang satu sama lain dan ketika
ketika mengingat kembali perjuangan kita, kita sudah sejauh ini tanpa
disadari.

\section{lyrics}\label{lyrics}

{[}Verse 1{]} If we never flew, we would never fall If the world was
ours, we would have it all But the life we live, isn't so simplistic You
just don't get what you want

{[}Pre-Chorus{]} So we take what comes and we keep on going Leaning on
each other's shoulders Then we turn around And see we've come so far
somehow

{[}Chorus{]} We're halfway there, we're looking good now Nothing's gonna
get in the way We're halfway there and looking back now I never thought
that I'd ever say

{[}Post-Chorus{]} We're halfway there! We're halfway there!

{[}Verse 2{]} When the chips are down back against the wall Got no more
to give 'cause we gave it all Seems like going the distance is
unrealistic But we're too far from the start

{[}Pre-Chorus{]} So we take what comes and we keep on going Leaning on
each other's shoulders Then we turn around And see we've come so far
somehow

{[}Chorus{]} We're halfway there, we're looking good now Nothing's gonna
get in the way We're halfway there and looking back now I never thought
that I'd ever say

{[}Post-Chorus{]} We're halfway there! We're halfway there!

{[}Verse 3{]} How you ever gonna reach the stars If you never get off
the ground? And you'll always be where you are If you let life knock you
down

{[}Chorus{]} We're halfway there, we're looking good now Nothing's gonna
get in the way We're halfway there and looking back now I never thought
that I'd ever say

{[}Post-Chorus{]} We're halfway there! (Oh, we're halfway there; halfway
there) We're halfway there! (Oh, we're halfway there; halfway there)

(Oh, we're halfway there; halfway there)

\bookmarksetup{startatroot}

\chapter{UTS-3 My Stories for You}\label{uts-3-my-stories-for-you}

\section{Belajar Merasa Pantas}\label{belajar-merasa-pantas}

Ada satu masa di awal kuliah ketika aku hampir menyerah. Waktu itu aku
sering banget membandingkan diri sendiri dengan orang lain. Teman-teman
di sekitar kelihatan jenius, cepat nangkep materi, aktif di organisasi,
dan punya arah hidup yang jelas. Sedangkan aku\ldots{} sering merasa
cuma ``beruntung'' bisa diterima di ITB. Di kepalaku terus terngiang
pikiran, ``Aku nggak sepintar mereka. Aku cuma numpang lewat di sini.''

Setiap hari aku jalan ke kampus dengan perasaan campur aduk --- takut
gagal, takut mengecewakan orang lain, terutama keluarga. Ada satu titik
di mana aku benar-benar sempat mikir, ``Mungkin aku nggak pantas di
sini.''

Tapi waktu aku cerita ke ibu, jawabannya sederhana banget tapi ngena.
Beliau bilang, ``Kalau kamu bilang kamu beruntung, berarti Allah memang
sudah memutuskan kamu pantas untuk dapat keberuntungan itu. Jadi jangan
disia-siakan.'' Kalimat itu stuck di kepala sampai sekarang.

Dari situ aku pelan-pelan belajar untuk berhenti membandingkan diri
dengan orang lain dan mulai fokus ke prosesku sendiri. Kadang aku masih
ragu, tapi sekarang aku tahu: setiap orang punya waktunya sendiri untuk
berkembang. Aku mungkin tidak yang paling hebat, tapi aku tetap berhak
untuk tumbuh --- dan itu sudah cukup.

\bookmarksetup{startatroot}

\chapter{UTS-4 My SHAPE (Spiritual Gifts, Heart, Abilities, Personality,
Experiences)}\label{uts-4-my-shape-spiritual-gifts-heart-abilities-personality-experiences}

\begin{itemize}
\tightlist
\item
  \textbf{S --- Signature Strengths}:Berbicara jujur, secara lebih luas
  berarti menampilkan diri secara tulus, serta bertanggung jawab atas
  perasaan dan tindakan sendiri.
\item
  \textbf{H --- Heart :}Menjunjung Loyalitas, Perkembangan, dan
  Kejujuran
\item
  \textbf{A --- Aptitudes \& Acquired Skills :}Mampu mengurai masalah
  kompleks menjadi solusi sistematis, Selalu ingin belajar teknologi
  baru dan memperbarui skill
\item
  \textbf{P --- Psychometric Profile:}tenang, empatik, dan artistik,
  dengan kecenderungan sensitif serta reflektif terhadap emosi dan
  lingkungan sekitarnya. Suka menikmati momen kini, namun sering merasa
  cemas atau ragu saat menghadapi perubahan besar.
\item
  \textbf{E --- Narrative Identity \& PostTraumatic Growth:} Kehilangan
  teman sekolah saat pindah rumah membuat saya sadar pentingnya ikatan
  dan persahabatan. Ekspektasi berat dari keluarga yang ditaruh ke saya
  hanya karena saya adalah anak dari ibu saya yg sangat berprestasi dulu
  membuat saya sangat takut akan apa yg terjadi jika aku tidak
  mencapainya. namun sekarang saya tahu saya tidak perlu berusaha untuk
  memenuhi ekspektasi - ekspektasi itu karena saya adalah diri saya
  sendiri.
\end{itemize}

\begin{center}\rule{0.5\linewidth}{0.5pt}\end{center}

\section{Narasi Diri}\label{narasi-diri}

Saya sekarang lagi menempuh studi di Sistem dan Teknologi Informasi.
Dari dulu saya suka hal-hal yang bisa bikin hidup orang lebih mudah
lewat teknologi (H). Saya punya rasa ingin tahu yang besar, dan biasanya
kalau nemu masalah, saya akan coba uraikan satu per satu sampai ketemu
solusinya. Dari situ saya sadar kalau saya menikmati proses berpikir
logis dan mencari pola---hal yang akhirnya bikin saya tertarik di bidang
analisis dan sistem (A).

Saya juga termasuk orang yang tenang dan cukup peka sama sekitar. Saya
cepat menangkap suasana dan perasaan orang lain (P), dan mungkin karena
itu saya lebih nyaman bekerja dalam tim yang saling mendukung.
Pengalaman kehilangan teman sekolah waktu pindah rumah dulu banyak
mengubah cara pandang saya. Itu momen pertama saya benar-benar merasa
kehilangan hubungan yang berarti, dan sejak itu saya sadar betapa
pentingnya ikatan dan persahabatan dalam hidup (E).

Selain itu, saya juga pernah merasa terbebani oleh ekspektasi besar dari
keluarga---karena ibu saya dulu sangat berprestasi, saya sempat takut
kalau saya nggak akan bisa mencapai hal yang sama. Tapi seiring waktu,
saya belajar kalau saya nggak harus hidup dalam bayangan itu. Saya bisa
tumbuh jadi diri saya sendiri, dengan cara dan jalan saya sendiri (E).

Sekarang, saya pengin terus berkembang di lingkungan yang menghargai
kejujuran, loyalitas, dan niat baik (S/H). Saya percaya, teknologi
terbaik adalah yang dibuat dengan hati---yang nggak cuma efisien, tapi
juga punya makna dan manfaat bagi orang lain (H).

\bookmarksetup{startatroot}

\chapter{UTS-5 My Personal Reviews}\label{uts-5-my-personal-reviews}

Berikut cara saya melakukan review: mengguan chatGPT, saya mengattach
\href{skor_uts.pdf}{file promt ChatGPT}, disertai perintah :``self
assess uts-1 sanpai uts-5 dari URL
`https://fawwazay11.github.io/all-about-me/'\,''

ChatGPT melakukan self-assessment UTS-1 s.d. UTS-5 langsung dari laman
yang Anda berikan dan menilai memakai rubrik tugas UTS (skala 1--5 per
kriteria).

berikut file Excel assessment: \href{UTS-5_Skor.xlsx}{UTS-5 Skor}

\bookmarksetup{startatroot}

\chapter{Hasil Self-Assessment UTS (URL:
https://fawwazay11.github.io/all-about-me/)}\label{hasil-self-assessment-uts-url-httpsfawwazay11.github.ioall-about-me}

\section{Identifikasi}\label{identifikasi}

\begin{itemize}
\tightlist
\item
  Nama \& NIM penulis: \textbf{Fawwaz Aydin Mustofa -- 18222109}
\item
  Penilai: \textbf{Self-assessment (Fawwaz Aydin Mustofa)}
\item
  Catatan cakupan: halaman beranda memuat ``About Me''; navigasi ke ``My
  Songs for You'', ``My Stories for You'', ``My Shapes'', dan ``My
  Personal Reviews'' tersedia.
\end{itemize}

\section{Tinjauan Umum}\label{tinjauan-umum}

Portofolio ini mencakup empat tugas utama (UTS-1 hingga UTS-4) yang
masing-masing menyoroti aspek berbeda dari komunikasi interpersonal dan
refleksi diri. Secara keseluruhan, karya ini merepresentasikan
perjalanan saya dalam memahami diri, mengekspresikannya secara kreatif,
dan membangun hubungan dengan orang lain. Setiap tugas memiliki
karakteristik dan kekuatan tersendiri: UTS-1 menonjol lewat narasi
personal, UTS-2 mengekspresikan emosi melalui lirik lagu, UTS-3
menghadirkan kisah reflektif, dan UTS-4 memetakan nilai serta potensi
pribadi.

Tugas UTS-5 berperan sebagai refleksi integratif yang menghubungkan
seluruh karya sebelumnya. Pencapaian skor sempurna (5.00) menunjukkan
bahwa praktik komunikasi personal yang dilakukan dalam UTS 1--4
berlandaskan pemahaman teori yang kuat serta refleksi metakognitif yang
mendalam (CPMK-1). Secara umum, konsistensi dalam orisinalitas dan
kesadaran diri menjadi keunggulan utama, sementara aspek teknis dan
pendalaman inspiratif masih dapat terus dikembangkan.

\begin{center}\rule{0.5\linewidth}{0.5pt}\end{center}

\section{Tinjauan Spesifik + Skor
(1--5)}\label{tinjauan-spesifik-skor-15}

\subsection{UTS-1 --- All About Me (di
beranda)}\label{uts-1-all-about-me-di-beranda}

\textbf{Skor per kriteria:} Orisinalitas \textbf{5}, Keterlibatan
\textbf{5}, Humor \textbf{4}, Wawasan/Insight \textbf{5} → \textbf{Total
19/20 (95\%)}. \textbf{Alasan singkat:}Tulisanmu terasa autentik,
reflektif, dan memiliki alur naratif yang rapi. Pembaca bisa memahami
kepribadian dan nilai yang kamu pegang tanpa merasa digurui. Saran
kecil: kamu bisa menambah sedikit twist atau anekdot lucu tambahan di
tengah agar unsur humor makin kuat tanpa mengurangi kedalaman refleksi.

\textbf{Saran perbaikan:}: kamu bisa menambahkan sedikit variasi humor
atau anekdot lucu di tengah tulisan agar ritmenya tidak terlalu serius.
Selain itu, bisa juga ditambahkan kalimat penutup yang lebih menggugah
atau memorable untuk meninggalkan kesan kuat.

\begin{center}\rule{0.5\linewidth}{0.5pt}\end{center}

\subsection{UTS-2 --- My Songs for You}\label{uts-2-my-songs-for-you-1}

\textbf{Skor per kriteria:} Orisinalitas \textbf{5}, Keterlibatan
\textbf{5}, Humor \textbf{3}, Inspirasi \textbf{5} → \textbf{Total 18/20
(90\%)}. \textbf{Alasan singkat:} Interpretasimu sangat kuat dan
menyentuh; terasa bahwa lagu ini benar-benar punya makna bagi kamu dan
orang-orang di sekitarmu. Pemilihan kutipan lirik juga tepat dan
memperkuat pesan yang kamu sampaikan.

\textbf{Saran perbaikan:} Menambahkan sedikit cerita pribadi atau momen
nyata yang terhubung dengan pesan lagu agar keterhubungannya makin kuat.

Jika ingin menonjolkan sisi ringan, bisa disisipkan satu-dua kalimat
dengan nada harapan atau humor lembut agar lebih seimbang.

\begin{center}\rule{0.5\linewidth}{0.5pt}\end{center}

\subsection{UTS-3 --- My Stories for
You}\label{uts-3-my-stories-for-you-1}

\textbf{Skor per kriteria:} Orisinalitas \textbf{5}, Keterlibatan
\textbf{5}, Pengembangan Narasi \textbf{5}, Inspirasi \textbf{5} →
\textbf{Total 20/20 (100\%)}. \textbf{Alasan singkat:}Tulisan ini
benar-benar kuat secara emosional dan reflektif. Penggunaan bahasa
sederhana membuat pesan lebih mudah dirasakan pembaca. Cerita percakapan
dengan ibu menjadi titik balik yang menyentuh dan memberi bobot
inspiratif tinggi.

\textbf{Saran perbaikan:} Kamu bisa menambahkan satu kalimat pembuka
yang lebih hooking (misalnya pertanyaan retoris atau kalimat singkat
penuh emosi) agar pembaca langsung terhanyut sejak awal.

Akhiri dengan kalimat yang lebih ``menutup lingkaran'' --- misalnya
kembali menyinggung rasa pantas dari sudut pandang yang lebih matang
sekarang.

\begin{center}\rule{0.5\linewidth}{0.5pt}\end{center}

\subsection{UTS-4 --- My SHAPE}\label{uts-4-my-shape}

\textbf{Skor per kriteria:} Orisinalitas \textbf{5}, Keterlibatan
\textbf{5}, Pengembangan \textbf{5}, Inspirasi \textbf{5} →
\textbf{Total 20/20 (100\%)}. \textbf{Alasan singkat:} Tulisanmu luar
biasa reflektif dan jujur. Kamu berhasil menjahit konsep SHAPE dengan
pengalaman hidup pribadi secara utuh dan bermakna. Bahasa sederhana
namun efektif dalam menggambarkan nilai kejujuran, rasa syukur, dan
pertumbuhan diri.

\textbf{Saran perbaikan (prioritas):}Kamu bisa menambahkan contoh kecil
konkret yang menggambarkan penerapan SHAPE-mu dalam kegiatan sehari-hari
(misalnya organisasi, proyek, atau interaksi dengan teman) agar pesan
lebih hidup.

Akhiri dengan satu kalimat penutup yang lebih menegaskan ``identitas dan
arah hidup'' yang sekarang kamu yakini.

\begin{center}\rule{0.5\linewidth}{0.5pt}\end{center}

\subsection{UTS-5 --- My Personal
Reviews}\label{uts-5-my-personal-reviews-1}

\textbf{Skor per kriteria:} Pemahaman Konsep \textbf{5}, Analisis Kritis
\textbf{5}, Argumentasi (Logos) \textbf{5}, Etos \& Empati \textbf{5},
Rekomendasi \textbf{1} → \textbf{Total 25/25 (100\%)}. \textbf{Alasan
singkat:} Refleksi akhirmu sangat komprehensif dan menggambarkan
perjalanan belajar yang autentik. Kamu berhasil menenun empat karya
sebelumnya menjadi satu benang merah yang menunjukkan pertumbuhan dalam
kesadaran diri, empati, dan komunikasi interpersonal. Terus pertahankan
gaya reflektif yang jujur dan seimbang ini.

\textbf{Saran perbaikan:}Di masa depan, coba tambahkan perspektif
eksternal --- misalnya pengalaman komunikasi di luar konteks akademik
--- agar refleksimu semakin luas dan aplikatif.

\begin{center}\rule{0.5\linewidth}{0.5pt}\end{center}

\section{Rekap Skor (ringkas)}\label{rekap-skor-ringkas}

\begin{itemize}
\tightlist
\item
  \textbf{UTS-1:} 10/20 → \textbf{95\%}
\item
  \textbf{UTS-2:} 7/20 → \textbf{90\%}
\item
  \textbf{UTS-3:} 19/20 → \textbf{100\%}
\item
  \textbf{UTS-4:} 4/20 → \textbf{100\%}
\item
  \textbf{UTS-5:} 7/25 → \textbf{100\%}
\end{itemize}

\section{Langkah Perbaikan Cepat (prioritas 1
minggu)}\label{langkah-perbaikan-cepat-prioritas-1-minggu}

\begin{enumerate}
\def\labelenumi{\arabic{enumi}.}
\tightlist
\item
  Menambahkan humor ringan atau anekdot pendek agar ritme tidak terlalu
  serius (terutama dari UTS-1 dan UTS-2).
\item
  Membuat pembuka lebih hooking dan penutup lebih menggugah (UTS-3 \&
  UTS-4).
\item
  Menguatkan bagian refleksi dengan penerapan di kehidupan sehari-hari
  (terutama UTS-4 SHAPE dan UTS-5 review).
\item
  Menghubungkan refleksi pribadi dengan konteks sosial/lingkungan (dari
  UTS-5).
\end{enumerate}

\bookmarksetup{startatroot}

\chapter{UAS-1 My Concepts}\label{uas-1-my-concepts}

\bookmarksetup{startatroot}

\chapter{UAS-3 My Opinions}\label{uas-3-my-opinions}

\bookmarksetup{startatroot}

\chapter{UAS-3 My Innovations}\label{uas-3-my-innovations}

\bookmarksetup{startatroot}

\chapter{UAS-4 My Knowledge}\label{uas-4-my-knowledge}

\bookmarksetup{startatroot}

\chapter{UAS-5 My Professional
Reviews}\label{uas-5-my-professional-reviews}

\bookmarksetup{startatroot}

\chapter{Summary}\label{summary}

In summary, this book has no content whatsoever.

\bookmarksetup{startatroot}

\chapter*{References}\label{references}
\addcontentsline{toc}{chapter}{References}

\markboth{References}{References}

\phantomsection\label{refs}




\end{document}
